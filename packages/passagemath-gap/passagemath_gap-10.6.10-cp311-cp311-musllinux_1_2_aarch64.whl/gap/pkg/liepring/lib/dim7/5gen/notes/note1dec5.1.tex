
\documentclass[12pt]{article}
%%%%%%%%%%%%%%%%%%%%%%%%%%%%%%%%%%%%%%%%%%%%%%%%%%%%%%%%%%%%%%%%%%%%%%%%%%%%%%%%%%%%%%%%%%%%%%%%%%%%%%%%%%%%%%%%%%%%%%%%%%%%%%%%%%%%%%%%%%%%%%%%%%%%%%%%%%%%%%%%%%%%%%%%%%%%%%%%%%%%%%%%%%%%%%%%%%%%%%%%%%%%%%%%%%%%%%%%%%%%%%%%%%%%%%%%%%%%%%%%%%%%%%%%%%%%
\usepackage{amsfonts}
\usepackage{amssymb}
\usepackage{sw20elba}

%TCIDATA{OutputFilter=LATEX.DLL}
%TCIDATA{Version=5.50.0.2890}
%TCIDATA{<META NAME="SaveForMode" CONTENT="1">}
%TCIDATA{BibliographyScheme=Manual}
%TCIDATA{Created=Thursday, June 27, 2013 16:28:19}
%TCIDATA{LastRevised=Sunday, July 28, 2013 17:11:05}
%TCIDATA{<META NAME="GraphicsSave" CONTENT="32">}
%TCIDATA{<META NAME="DocumentShell" CONTENT="Articles\SW\mrvl">}
%TCIDATA{CSTFile=LaTeX article (bright).cst}

\newtheorem{theorem}{Theorem}
\newtheorem{axiom}[theorem]{Axiom}
\newtheorem{claim}[theorem]{Claim}
\newtheorem{conjecture}[theorem]{Conjecture}
\newtheorem{corollary}[theorem]{Corollary}
\newtheorem{definition}[theorem]{Definition}
\newtheorem{example}[theorem]{Example}
\newtheorem{exercise}[theorem]{Exercise}
\newtheorem{lemma}[theorem]{Lemma}
\newtheorem{notation}[theorem]{Notation}
\newtheorem{problem}[theorem]{Problem}
\newtheorem{proposition}[theorem]{Proposition}
\newtheorem{remark}[theorem]{Remark}
\newtheorem{solution}[theorem]{Solution}
\newtheorem{summary}[theorem]{Summary}
\newenvironment{proof}[1][Proof]{\noindent\textbf{#1.} }{{\hfill $\Box$ \\}}
\input{tcilatex}
\addtolength{\textheight}{30pt}

\begin{document}

\title{Descendants of 5.1 of order $p^{7}$}
\author{Michael Vaughan-Lee}
\date{June 2013}
\maketitle

This is the same calculation as is needed in the calculation of \ algebra
6.62. See note6.62 in the $p^{6}$ directory. We have two parameters $x,y$,
where $x,y$ are integers with $y\neq 0\func{mod}p$. Parameter pairs $(x,y)$
and $(z,t)$ give isomorphic algebras if and only if%
\[
\left( 
\begin{array}{ll}
1 & 0 \\ 
z & t%
\end{array}%
\right) =\left( 
\begin{array}{ll}
\mu  & \nu  \\ 
\omega \nu  & \mu 
\end{array}%
\right) \left( 
\begin{array}{ll}
1 & 0 \\ 
x & y%
\end{array}%
\right) \left( 
\begin{array}{ll}
\mu +\nu x & \nu y \\ 
\omega \nu y & \mu +\nu x%
\end{array}%
\right) ^{-1}\func{mod}p
\]%
for some matrix $\left( 
\begin{array}{ll}
\mu  & \nu  \\ 
\omega \nu  & \mu 
\end{array}%
\right) $ with determinant coprime to $p$. (Here, as elsewhere, $\omega $ is
a primitive element modulo $p$.) So we need to compute representatives for
the orbits of non-singular matrices $\left( 
\begin{array}{ll}
1 & 0 \\ 
x & y%
\end{array}%
\right) \in \,$GL$(2,p)$ under the action of the group of non-singular
matrices $\left( 
\begin{array}{ll}
\mu  & \nu  \\ 
\omega \nu  & \mu 
\end{array}%
\right) \in \,$GL$(2,p)$ given above. There are $p$ orbits.

It is easy enough to generate the $p$ orbit representatives with a simple
loop over all non-singular matrices $\left( 
\begin{array}{ll}
\mu & \nu \\ 
\omega \nu & \mu%
\end{array}%
\right) $ and $\left( 
\begin{array}{ll}
1 & 0 \\ 
x & y%
\end{array}%
\right) $. However this method has complexity $p^{4}$ for output of size $p$%
, which is not very satisfactory! Can we do better? Multiplying $\left( 
\begin{array}{ll}
\mu & \nu \\ 
\omega \nu & \mu%
\end{array}%
\right) $ through by a non-zero constant has no effect on the action, so we
can assume that $\mu =0,1$, and that if $\mu =0$ then $\nu =1$. This reduces
the complexity to $p^{3}$.

\end{document}
